% !TeX root = RJwrapper.tex
\title{A Backbone for Simulating Quantities of Interest from Generalized Linear
Models}
\author{by Christopher Gandrud}

\maketitle

\abstract{%
Simulating quantities of interest estimated from generalized linear
models (GLM) can be an effective way to explore and communicate
statistical results. \pkg{coreSim} provides core functions that can
serve as a reliable and extensible ``backbone'' to new packages for
simulating quantaties of interest for finding and plotting simulated
quantities of interest from GLMs. I demonstrate how to use \pkg{coreSim}
as a backbone by showing how it is incorporated into the new
\pkg{pltesim} package for simulating and displaying probabilistic
long-term effects in models with temporal dependence.
}

\subsection{Introduction}\label{introduction}

There has been a recent trend in the social sciences to improve the
interpretation of results from generalized linear models (GLM) by
presenting estimated quantities of interest from substantively
meaningful scenarios
\citep[e.g.][]{gandrud2005,Licht2011,WilliamsWhitten2012}.
\citet{King2000} advanced a post-estimation simulation technique for
estimating the uncertainty around these estimates. A number of R
packages implement this approach {[}CITE{]}. In particular, the
\CRANpkg{Zelig} package \citep{R-Zelig} implements this technique for a
wide range of model types in an attempt to be ``everyone's statistical
software''.

While all of these packages based on the same simulation procedure, all
of these packages rely on different, custom built ways of doing these
simulations. This presents unnecessarily increases the labor needed to
use post-estimation simulations for showing results in new modelling
situations. Current implementations of these packages have also proven
to be unreliable in ways directly related to their architecture.

In this paper I introduce a new R package--\CRANpkg{coreSim}--that aims
to be an extensible and reliable ``backbone'' for future analyses and R
packages that find post-estimation simulations for substantively
relevant scenarios. I give an example of this by showing how
\pkg{coreSim} forms the basis of the new \pkg{pltesim} package for
simulating and displaying probabilistic long-term effects in models with
temporal dependence.

\subsection{Post-estimation simulations from
GLMs}\label{post-estimation-simulations-from-glms}

One way to estimate substantively meaningful quantities of interest for
complex scenarios from with their associated uncertainty is through
post-estimation
simulations.\footnote{See \citet{King2000} for a comparision with related fully Bayesisan Markov chain Monte Carlo and Bootstrapping methods.}
The procedure is straightforward. Remember that a generalized linear
models can be summarized by two equations. One describes the stochastic
component:

\begin{equation}
Y_{i} \sim f(\theta_{i}, \alpha)
\end{equation}

\noindent Here the dependent variable \(Y\) is a random drawn from the
probability density function \(f(\theta_{i}, \alpha)\). The other
equation describes the systematic component:

\begin{equation}
\hspace{1cm} \theta_{i} = g(X_{i}, \beta),
\end{equation}

where \(X\) is a vector of explanatory variables and \(\beta\) is a
vector of effect parameters. The function \(g\) is often referred to as
the link function, which specifies how the variables and effect
parameters are translated into \(\theta\).

We can use post-estimation simulations to find and communicate
substantively meaningful results from these models. To do this we first
estimate \(\hat{\beta}\) and \(\hat{\alpha}\). Second, we drawn \(n\)
number of values of these parameters from the multivariate normal
distribution using the parameter point estimates and their variance
covariance matrix with a mean of \(\hat{\gamma}\)--a vector created by
stacking \(\hat{\beta}\) and \(\hat{\alpha}\)--and variance of
\(\hat{V(\hat{\gamma})}\):

This is straightforward using two functions from the \CRANpkg{stats}
package--\code{coef} and \code{vcov}--as well as the \code{mvrnorm}
function from the \CRANpkg{MASS} package. Both \CRANpkg{stats} and
\CRANpkg{MASS} are included with the default R installation. For
example:

\begin{Schunk}
\begin{Sinput}
# Load package that contains the Prestige data set
library(car)

# Estimate normal linear model
m1 <- lm(prestige ~ education + type, data = Prestige)

# Extract 
\end{Sinput}
\end{Schunk}

\subsection{Previous post-estimation simulation
packages}\label{previous-post-estimation-simulation-packages}

The \CRANpkg{Zelig} package could potentially help streamline this
process and in so doing form the basis of R packages that generate
post-estimation simuliations in novel areas.

\CRANpkg{Zelig} has a number of issues that have made it unreliable over
time. By aiming to be everyone's statistical software, \CRANpkg{Zelig}
relies on many (13 imported) dependencies.

\subsection{Summary}\label{summary}

\bibliography{gandrud}

\address{%
Christopher Gandrud\\
City University London and Hertie School of Governance\\
Rhind Building\\ London, EC1V 0HB\\
}
\href{mailto:christopher.gandrud@city.ac.uk}{\nolinkurl{christopher.gandrud@city.ac.uk}}

